\chapter{Context \& State of the Art}

\section{Credit Scoring Fundamentals}
Credit scoring is a statistical analysis performed by lenders and financial institutions to access a person's creditworthiness. Lenders use credit scoring to decide on whether to extend or deny credit. Use of credit scoring allows for objective analysis of risk factors.
Historically, this was done using the "5 Cs of Credit": Character, Capacity, Capital, Collateral, and Conditions. In the algorithmic era, these become features in a dataset.

\section{The Imbalance Challenge}
In credit risk modeling, the "Positive" class (Defaulters) is always the minority. A dataset might have 95\% good payers and 5\% defaulters. Use of standard accuracy metrics is dangerous here; a trivial model predicting "Good" for everyone would be 95\% accurate but 100\% useless.
\textbf{Synthetic Minority Over-sampling Technique (SMOTE)} addresses this by creating synthetic line segments joining k-nearest neighbors of the minority class, effectively "hallucinating" new plausible defaulters to train the model more robustly.

\section{Explainable AI (XAI)}
\subsection{The "Right to Explanation"}
Article 22 of the GDPR states that individuals have the right not to be subject to a decision based solely on automated processing. Valid explanations are required.

\subsection{SHAP (SHapley Additive exPlanations)}
We utilize SHAP values, a concept from Game Theory. It assigns each feature an importance value for a particular prediction.
$$ \phi_i(f,x) = \sum_{z' \subseteq x'} \frac{|z'|! (M - |z'| - 1)!}{M!} [f_x(z') - f_x(z' \setminus i)] $$
In simple terms, SHAP calculates the marginal contribution of a feature (e.g., "Income") to the final score, averaged over all possible combinations of other features. This provides a mathematically consistent explanation used in our API.

\section{Generative AI in Fintech}
The emergence of Large Language Models (LLMs) like GPT-4 and Gemini allows for a new layer of interaction. Instead of static dashboards, we can now offer "Conversational Banking."
In this project, GenAI is not used for the risk score itself (due to hallucination risks) but for the \textbf{interpretation layer}—taking the hard data and SHAP values and crafting a human-readable, empathetic explanation for the user.

\section{The Shift to MLOps}
Google's paper "Hidden Technical Debt in Machine Learning Systems" highlighted that ML code is only a tiny fraction of a real-world system. Storage, ingestion, monitoring, and serving infrastructure make up the bulk.
\textbf{MLOps (Machine Learning Operations)} is the discipline of applying DevOps principles to ML systems. It aims to unify ML system development (Dev) and ML system operation (Ops).
Key components include:
\begin{itemize}
    \item \textbf{Data Versioning:} Treating data like code (DVC).
    \item \textbf{Model Registry:} Formalizing model stages (MLflow).
    \item \textbf{Continuous Training (CT):} Automatically retraining when data drifts.
\end{itemize}
