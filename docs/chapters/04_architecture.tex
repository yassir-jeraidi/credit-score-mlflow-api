\chapter{Architecture \& Implementation}

\section{Global Architecture}
The system enables a seamless workflow from Data to Deployment. It consists of three main layers:
\begin{enumerate}
    \item \textbf{Data Layer:} Storage and versioning of datasets (managed via DVC).
    \item \textbf{Model Layer:} ML training pipelines, experiment tracking (MLflow), and model serialization.
    \item \textbf{Serving Layer:} An API exposing the model to the outside world.
\end{enumerate}

\begin{figure}[H]
    \centering
    \includegraphics[width=0.95\textwidth]{architecture.jpeg}
    \caption{System Architecture Diagram}
    \label{fig:architecture}
\end{figure}

\section{Backend \& API Controller}
The backend is built using \textbf{FastAPI} (Python), chosen for its high performance and automatic documentation generation (Swagger UI). It acts as the controller, receiving client requests, validating data schemas using \textbf{Pydantic}, and delegating prediction tasks to the ML engine.
While frameworks like Spring Boot are robust, Python's native support for our ML stack made a Python-based backend the logical choice for seamless integration.

\section{ML Engine & Serialization}
The trained model is serialized (pickled) and loaded into memory upon API startup. This ensures low-latency predictions. We use MLflow to manage the model registry, allowing us to stage and promote specific model versions (e.g., Staging $\rightarrow$ Production) without changing the application code.

% Example of referencing other system diagrams if available
\begin{figure}[H]
    \centering
    \includegraphics[width=0.9\textwidth]{docker-containers.png}
    \caption{Containerized Environment (Docker Services)}
    \label{fig:docker}
\end{figure}
